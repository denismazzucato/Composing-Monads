\section{Monadic Evaluator}
\label{monadic_evaluator}

Ora che sappiamo cosa siano i \textit{Monads} possiamo applicari al primo
esempio nella sezione \ref{a_simple_evaluator}, in questa sezione costruiremo un
valutatore monadico per rappresentare le caratteristiche volute.
Senza entrare troppo nel dettaglio divido l'esempio in due parti per semplicità,
nelle prossime sezione vedremo come combinare il tutto in maniera semplice ed
elegante con un sistema generalizzabile.

\subsection{Exceptions}
\label{exceptions}
Per rappresentare la gestione delle eccezioni definisco il \textit{Monad}

\begin{figure}[H]
  \centering
  \footnotesize % adapt to your page size
  \begin{haskellsyntax}
  \textbf{data}\ Exception\ a &= Raise\ String\ |\ Return\ a \\\\

  map &:: (a \to b) \to Exception\ a \to Exception\ b\\
  map\ f\ (Raise\ s) &= Raise\ s\\
  map\ f\ (Return\ x) &= Return\ (f\ x)\\\\

  unit &:: a \to Exception\ a\\
  unit &= Return\\\\

  join &:: Exception\ (Exception\ a) \to Exception\ a\\
  join\ (Raise\ s) &= Raise\ s\\
  join\ (Return\ x) &= x
  \end{haskellsyntax}%
\end{figure}
La dimostrazione la si può trovare in appendice \ref{exception_is_a_monad}.
Definiamo quindi il corrispondente interprete per la grammatica
\ref{abstract_syntax}
\begin{figure}[H]
  \centering
  \footnotesize % adapt to your page size
  \begin{haskellsyntax}
  eval &:: Expr \to Env \to Exception\ Value\\
  eval \ (Const \ v) \ env &= unit\ v\\
  eval \ (Var \ x) \ env   &= lookup \ env \ x\\
  eval \ (e_0 + e_1) \ env &= bind\ (eval\ e_0\ env)\\
                           & \qquad \lambda\ v_0 \to bind\ (eval\ e_1\ env)\\
                           & \qquad \qquad \lambda\ v_1 \to unit (v_0 + v_1)
  \end{haskellsyntax}%
\end{figure}
Nota se \textit{Exception }è \textit{Monad} posso usare sia join che bind,
per questo esempio è comodo utilizzare il bind operator.
Ora passiamo ad osservare i pregi di questo esempio, notiamo subito come il
codice sia più coinciso e descriva meglio l'intenzione piuttosto che la
computazione.
L'utilizzo di \textit{unit} e \textit{bind} semplifica sia l'implementazione che
il ragionamento sull'interprete rendendolo più resistente a possibili errori di
programmazione.

\subsection{Output}
\label{output}

Per l'output definiamo il tipo astratto \textit{Writer} che definisce un
meccanismo per tenere la traccia del log prodotto (se prodotto), richiedo
inoltre che il tipo di

\begin{figure}[H]
  \centering
  \footnotesize % adapt to your page size
  \begin{haskellsyntax}
    \textbf{data}\ Writer\ s\ a &= (s,\ a) \\\\

      map &:: Monoid\ s \Rightarrow (a \to b) \to Writer\ s\ a \to Writer\ s\ b\\
      map\ f\ (s,\ a) &= (s,\ f\ a)\\\\

      unit &:: Monoid\ s \Rightarrow a \to Writer\ s\ a\\
      unit\ x&= (zero,\ x)\\\\

      join &:: Monoid\ s \Rightarrow Writer\ s\ (Writer\ s\ a) \to Writer\ s\ a\\
      join\ (s',\ (s'',\ a)) &= (s' \doubleplus s'',\ a)
  \end{haskellsyntax}%
\end{figure}

Definiamo, come per le eccezioni, la funzione per la grammatica
\ref{abstract_syntax}

\begin{figure}[H]
  \centering
  \footnotesize % adapt to your page size
  \begin{haskellsyntax}
  eval &:: Expr \to Env \to Writer\ [String]\ Value\\
  eval \ (Const \ v) \ env &= unit\ v\\
  eval \ (Var \ x) \ env   &= \textbf{case}\ lookup \ env \ x\ \textbf{of}\\
                           & \qquad Raise\ s \to unit\ defaultValue\\
                           & \qquad Ok\ x \to unit\ x\\
  eval \ (e_0 + e_1) \ env &= bind\ (eval\ e_0\ env)\\
                           & \qquad \lambda\ v_0 \to bind\ (eval\ e_1\ env)\\
                           & \qquad \qquad \lambda\ v_1 \to unit (v_0 + v_1)\\
  eval \ (Trace\ s\ e)\ env  &= ([s],\ eval\ e\ env)
  \end{haskellsyntax}%
\end{figure}

Per il caso delle eccezioni \ref{exceptions}, i
\textit{Monads} portano gli stessi benefici osservati precedentemente.
Unica nota negativa, in questo caso per l'espressione $Var\ x$ bisogna
controllare che la variabile sia stata definita, non gestendo le eccezioni non
abbiamo alcun modo di ritornare un caso di errore quando la variabile non è
definita.

\pagebreak

\subsection{Observations on Monadic Evaluator}
\label{observations_on_monadic_evaluator}
Da questi due esempi notiamo subito qualche osservazione.

\paragraph*{Why Composing Monads}
\label{why_composing_monads}
Per questa sezione \ref{monadic_evaluator}
ho dovuto dividere in due il semplice interprete concreto.
Questa necessità è dovuta dal fatto che il monade delle eccezioni definisce un
meccanismo per gestire gli errori, mentre il monade per la traccia dell'output
gestisce i logs.
Quindi non abbiamo raggiunto lo scopo finale, i due interpreti non sono, se
presi singolarmente, equivalenti al primo \ref{a_simple_evaluator}.
Possono definirsi "equivalenti" solamente se ne prendiamo l'unione degli
effetti.\newline

Per risolvere questo problema ci troviamo davanti a due strade, la prima
possibilità consiste nel definire un nuovo \textit{Monad} caratterizzato dal
tipo
\begin{align*}
  \textbf{data}\ M\ s\ a &= Raise\ String\ |\ Return\ (s,\ a)
\end{align*}
Questo tipo astratto però è definito ad-hoc e non presenta nessuna
caratteristica di generalizzazione, implica che la prossima feature da
aggiungere renderà inutilizzabile il tipo $M$ e necessiterà di un nuovo tipo
$M'$ specializzato anch'esso sulla nuovo task.\\
La seconda possibilità suppone l'esistenza di un compositore di \textit{Monads}
\begin{align*}
  \textbf{data}\ MonadCompositor\ m_0\ m_1\ a = MC\ m_0\ m_1\ a
\end{align*}
Anche $MonadCompositor$ è un \textit{Monad} e passati due monadi come parametri
ne ritorna la composizione naturale.
Con questo framework potremmo definire l'interprete sul tipo
\begin{align*}
  eval :: Expr \to Env \to MonadCompositor\ Exception\ (Writer\ [String])
\end{align*}
Con il vantaggio di dover modificare solamente poca sintassi dell'interprete
monadico già definito.
Inoltre se una nuova feature dovesse essere aggiunta lo sforzo per raggiungere
il task sarebbe minimo utilizzando il tipo
\begin{align*}
  MonadCompositor\ Exception\ (MonadCompositor\ FeatureMonad\ (Writer\ [String]))
  \end{align*}
Data la sola definizione del $FeatureMonad$ che dovrà solamente occuparsi di
implementare il proprio \textit{Monad} in un ambiente chiuso.\newline

Se questa supposizione fosse vera avremmo un meccanismo per modularizzare
questi \textit{Monads} e combinarli a piacimento tra loro.
Vedremo come questa supposizione non sia corretta in maniera generale nel
capitolo \ref{composing_monads}, difatti il \textit{Monad Composer} richiede
alcune assunzioni e requisiti che vedremo più avanti.

\pagebreak
\paragraph*{Monad Comprehensions}
\label{monad_comprehensions}

Scrivendo le funzioni di valutazioni ci accorgiamo di come la sintassi sia
accessibile solamente da chi abbia piena conoscenza del funzionamento delle
monadi, quindi ad un lettore esterno il codice di $eval$, in chiave monadica,
risulta comunque complesso.
Per migliorare la qualità del codice sotto l'aspetto della comprensibilità ci
basta notare che se $M$ è un \textit{Monad} possiamo definire le sue operazioni
con la \textit{comprehension notation}
\footnote{la \textit{comprehension notation} è un espressione $[\ exp\ |\ gs\ ]$
dove $exp$ è un espressione e $gs$ è chiamato "generatore".}
semplificando così la definizione di metodi basati su $map$, $unit$ e $join$.\\
Traduciamo quindi queste operazioni nella forma di
\textit{comprehension notation}

\begin{align*}
  {[\ exp\  |\  x \leftarrow e\ ]} &= map\ (\lambda\ x \to exp)\ e &&mapComp\\
  {[ \  exp \  ]} &= unit \  exp &&unitComp\\
  {[\ exp\ |\ gs,\ hs\ ]} &= join\ {[\ {[\ exp\ |\ hs]}\ |\ gs\ ]} &&joinComp\\
\end{align*}

Per convenienza possiamo derivare le regole $(M\ 1\dots7)$ e definirne di nuove
con questa notazione

\begin{align*}
  {[\ x\  |\  x \leftarrow xs\ ]} &= xs &&compId\\
  {[\ f\ x\ |\ x \leftarrow map\ g\ e\ ]} &= {[\ f\ (g\ x)\ |\ x \leftarrow e\ ]} &&compMap\\
  {[\ f\ x\ |\ x \leftarrow unit\ e\ ]} &= {[\ f\ e\ ]} &&compUnit\\
  {[\ exp\ |\ x \leftarrow join\ e\ ]} &= {[\ exp\ |\ z \leftarrow e,\ x \leftarrow z\ ]} &&compJoin
\end{align*}

Queste possono quindi essere usate per definire l'operatore $join$, vedi
\begin{align*}
  &\qquad join\ e &&\\
  &=\quad {[\ x\ |\ x \leftarrow e\ ]} &&\qquad(compId)\\
  &=\quad {[\ x\ |\ z \leftarrow e,\ x \leftarrow z\ ]} &&\qquad(compJoin)
\end{align*}

Se applichiamo questa sintassi alla funzione di valutazione otteniamo il
seguente risultato (solo per il caso in cui ho un'espressione del tipo somma,
secondo me l'unico caso da migliorare)

\begin{align*}
  eval\ (e_0 + e_1)\ env &= {[\ v_0 + v_1\ |\
                            v_0 \leftarrow eval\ e_0\ env,\
                            v_1 \leftarrow eval\ e_1\ env\ ]}
\end{align*}

La dimostrazione:
\begin{align*}
  &{[\ v_0 + v_1\ |\ v_0 \leftarrow eval\ e_0\ env,\ v_1 \leftarrow eval\ e_1\ env\ ]} = \\
  &bind\ (eval\ e_0\ env) (\lambda\ v_0 \to bind\ (eval\ e_1\ env) (\lambda\ v_1 \to unit (v_0 + v_1)))
\end{align*}
è stata svolta e riportata in appendice \ref{monad_comprehensions_equivalence}.
Questa nuova versione in stile \textit{Monad comprehension} è valida per
entrambi i codici, sia per l'output che per le eccezioni.

\vfill % for the footnote
\pagebreak