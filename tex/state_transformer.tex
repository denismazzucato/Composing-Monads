\label{state_transformer}

In questa sezione mostro il motivo per cui non sempre la composizione ha
l'effetto sperato.
Il $Monad$ per lo $State\ Transformer$ difatti rappresenta un esempio pratico
in cui, dato un $Monad M$, la composizione generalizzata non è ciò che ci
aspettiamo
\begin{align*}
  \dots
\end{align*}
Una spiegazione a questo fatto è che la composizione di monadi generata dal
nostro framework non riesce ad esprimere la relazione che intercorre tra i
componenti del monade composto.
Questa spiegazione, istanziata al nostro caso, in particolare non esprime
l'equivalenza tra il parametro in input al $Reader\ Monad$ con l'output prodotto
nel $Writer\ Monad$.\newline

