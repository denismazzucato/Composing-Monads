\section*{Abstract}

I \textit{Monads} sono un'importante strumento per i linguaggi funzionali.
Differenti \textit{Monads} possono essere utilizzati per modellare
un ampio gruppo di features di programmazione.
Vedremo come questi meccanismi portino benefici in un semplice esempio.
Molto spesso progetti di qualsiasi entità richiedono combinazioni di queste
feature, è quindi importante avere tecniche per poter combinare varie
\textit{Monads} in un singolo \textit{Monad}.
In pratica questa non è un'operazione difficile, infatti è
possibile combinare qualche feature specifica per costruire nuovi
\textit{Monads} con facilità.
Le tecniche solitamente utilizzate però sono generalmente ad-hoc e sembra molto
più difficile il problema di trovare una caratterizzazione generale per
combinare \textit{Monads} in maniera arbitraria.
Vedremo poi tre costrutti generali per comporre \textit{Monads}, ognuno di esso
collegato alle proprie condizioni e funzioni ausiliarie.
Concludendo con un esempio pratico.