\section{Monads}
\label{monads}

Walder\cite{walder0} definisce un \textit{Monad} come un tipo costrutture unario
\textit{M}, con un unico parametro, insieme alle tre funzioni \textit{map},
\textit{unit} e \textit{join} definite dal tipo

\begin{align*}
  map  &:: (a \to b) \to (M \ a \to M \ b) \\
  unit &:: a \to M \ a\\
  join &:: M\ (M \ a) \to M \ a
\end{align*}

Altre caratterizzazioni (equivalenti) possono essere definite per il concetto
dei \textit{Monads}, per esempio come definito in Walder\cite{walder1}

\begin{align*}
  bind :: M \ a \to (a \to M \ b) \to M \ b
\end{align*}

Ragionando su questa forma è semplice notare come sia del tutto equivalente alle
precedenti mediante le relazioni
\begin{align*}
  bind \ x \ f &= join \ (map \ f \ x)\\
  join\ x &= bind\ x\ id
\end{align*}

In ogni caso, per questo report, ho scelto di lavorare con \textit{map},
\textit{unit} e \textit{join} perchè rendono più marcata la differenza tra
\textit{Monads} e \textit{Premonads} e inoltre rendono più
semplice lo svolgimento delle dimostrazioni in questo framework.\newline

In aggiunta a queste funzioni si richiede di soddisfare una collezione di
proprietà algebriche

\begin{align*}
    map \ id &= id \law{M1} \\
    map \ f \ .\ map \ g &= map\ (f\ .\ g) \law{M2}\\
    unit\ .\ f &= map\ f\ .\ unit \law{M3}\\
    join\ .\ map\ (map\ f) &= map\ f\ .\ join \law{M4}\\
    join\ .\ unit &= id \law{M5}\\
    join\ .\ map\ unit &= id \law{M6}\\
    join\ .\ map\ join &= join\ .\ join \law{M7}
\end{align*}

Per lo scopo del report è conveniente definire per parti un \textit{Monad};
se \textit{M} è un tipo costruttore unario diremo che
\begin{enumerate}
  \item \textit{M} è un \textit{Functor} se c'è una funzione
    $map  :: (a \to b) \to (M \ a \to M \ b) $ che soddisfa le leggi
    \textit{(M1)} e \textit{(M2)}.
  \item \textit{M} è un \textit{Premonad} se c'è una funzione
    $unit :: a \to M\ a$ che soddisfa la legge \textit{(M3)}.
  \item \textit{M} è un \textit{Monad} se c'è una funzione
    $join :: M (M\ a) \to M\ a$ che soddisfa le leggi \textit{(M4)},
    \textit{(M5)}, \textit{(M6)} e \textit{(M7)}.

\end{enumerate}

\subsection*{Examples}

Mostriamo come due linguaggi di programmazione (non sperimentali) definiscano i concetti
di $Monads$.

\paragraph*{Haskell}
Per Haskell un Monade è definito in modo molto simile a $Gofer$, ovvero è un
oggetto con le proprietà
\begin{itemize}
  \item $return$, corrispondente a $unit$
  \item $bind$, corrispondente a $join$, vedi l'equazione $bind\ x\ f = join\ (map\ f\ x)$
\end{itemize}

\paragraph*{Scala}
Per scala invece un Monade è definito con le seguenti proprietà
\begin{itemize}
  \item $unit$
  \item $flatMap$, corrispondente a $join$, vedi $flatMap$ è l'operatore $bind$
\end{itemize}
\pagebreak